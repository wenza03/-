\documentclass{article}
\usepackage[UTF8]{ctex}
\usepackage{anyfontsize}
\usepackage{amsmath,mathtools}
\usepackage[a4paper, total={6in, 8in}]{geometry}
\usepackage{float}
\usepackage{graphicx}
\usepackage[ruled,linesnumbered]{algorithm2e}
\usepackage[usenames,dvipsnames]{xcolor}
\usepackage{listings,color}
\usepackage{tikz}
\definecolor{shadecolor}{RGB}{241, 241, 255}
\newcounter{problemname}
\newenvironment{problem}{\stepcounter{problemname}\par\noindent\textbf{题目\arabic{problemname}的解答. }}{\par}

\definecolor{mypurple}{rgb}{186,85,211}
\definecolor{mygreen}{rgb}{0,0.6,0}
\definecolor{mygray}{rgb}{0.5,0.5,0.5}
\definecolor{mymauve}{rgb}{0.58,0,0.82}
\geometry{a4paper,left=2cm,right=2cm,top=3cm,bottom=3cm}
\lstset{
    language=C++,
    backgroundcolor=\color[RGB]{245,245,244},
    basicstyle          =   \sffamily,          % 基本代码风格
    keywordstyle        =   \color{ purple},         % 关键字风格
    numberstyle=            \tiny\color{codegray},
    commentstyle        =   \rmfamily\itshape,  % 注释的风格,斜体
    stringstyle         =   \ttfamily,  % 字符串风格
    flexiblecolumns,                % 别问为什么,加上这个
    numbers             =   left,   % 行号的位置在左边
    showspaces          =   false,  % 是否显示空格,显示了有点乱,所以不现实了
    numberstyle         =   \zihao{-5}\ttfamily,    % 行号的样式,小五号,tt等宽字体
    showstringspaces    =   false,
    captionpos          =   t,      % 这段代码的名字所呈现的位置,t指的是top上面
    frame               =   lrtb,   % 显示边框
}
\title{\textbf{算法设计与分析课程作业 作业6}}
\author{162140224 刘昊文}
\date{\today}

\begin{document}
\maketitle
\section{$Dijkstra$负边问题}
\begin{tikzpicture}
  
    \draw 


    (12, 5) node[circle, black, draw](p){2}
    (18, 1) node[circle, black, draw](g){3}
    (10, 1) node[circle, black, draw](n){1}
    (14, 0) node[circle, black, draw](d){node};
    
    
    \draw[-] (g) -- node[above] {4}(n);
    \draw[-] (n) -- node[above] {3}(p);
    \draw[-] (p) -- node[above] {-5}(g);

\end{tikzpicture}
    

    Dijkstra算法采用了贪心策略,假设源点为1点,此时dist[]数组为{0,4,3},这时,我们会选择最近的2节点
    进行更新,此时dist[3]>dist[2]+(-5),所以dist[3]=dist[2]-5
    得到的dist为{0,3,-2},算法得出2结点的最短路答案是3,事实上,2节点的最短路径应该为1->3->2,
    为-2,但由于2被选择后就无法被更新,导致了算法
    在处理负权值时的错误,原因就在于对于距离较大的一个点,加上负权值边后,可能距离小于已经确定的最近点,而由于最近点已被
    选择且无法更新,所以产生了错误。

\newpage
    \section{钱币兑换问题}

对于这道题,硬币的面值很特殊,都是2的n次幂,与兑换的值的二进制数存在关系
    ,所以,对于需要兑换的值将其转换成二进制,如果第t位(最低位算作第1位)上的二进制值为1
    ,则存在一个$2^{t-1}$的硬币组成答案。
\begin{problem}
    \begin{algorithm}[!htbp]
        \SetAlgoLined
        \SetKwProg{Alg}{Algorithm Start}{}{}
        \SetKwProg{Fn}{Function}{}{}
        
        \KwIn{the number of money $\mathbf{mon}$}
        \KwOut{the number of conins $\mathbf{tot}$
            \newline the coins array of money $\mathbf{ansCoins}$}
        
        \Fn{getCoins$(\mathbf{mon})$}{
            $nowCoins \leftarrow 1$
            \newline
            $tot \leftarrow 0$
            \newline
            \While{$mon\geq 1$}
            {
                \If{$mon \% 2 =1 $}
                {
                    
                    $tot \leftarrow tot+1$
                    \newline
                    $ansCoins[tot] \leftarrow nowCoins$
                }
            }    
                $nowCoins \leftarrow nowCoins*2$\newline
                $mon \leftarrow mon/2$
                \newline
                \Return{$tot,ansCoins$}
            }
        \textbf{Function End}\\
        \caption{Coins of Money}
    \end{algorithm}

    
\end{problem}



\newpage
\section{$Kruskal$}
$Kruskal$算法中利用了并查集避免成环,当边两个结点的并查集根节点相同时,则说明加入
这条边会使图形成环,
需要跳过这条边。
    \begin{lstlisting}[language=C++,numbers=left,breaklines=true]
int find(int x)//并查集返回节点x的根节点并进行路径压缩
{
    if(fa[x]==x) return x;
    else return find(fa[x]);
}
struct node{//边集合
    int from,to,valve;
}ma[200005];
bool cmp(node a,node b)//排序使边权小的靠前
{
    return a.valve<b.valve;
}
int main(){
    int ans=0,tot=0,t1,t2,t3,n,m,s;
    cin>>n>>m;
    for(int i=1;i<=m;i++)
        cin>>ma[i].from>>ma[i].to>>ma[i].valve;
    sort(ma+1,ma+1+m,cmp);
    for(int i=1;i<=n;i++)//初始化并查集
    fa[i]=i;
    for(int i=1;i<=m;i++)
    {
        int t1=find(ma[i].from),t2=find(ma[i].to);//找出当前处理边两个端点的并查集根节点
        if(t1==t2)//相等则说明如果相连则会称号,需要跳过这条边
        continue;
        fa[t1]=t2;//标记好并查集
        ans+=ma[i].valve;//更新答案
        tot++;//边数加一
        if(tot==n-1)//最小生成树边数为点数-1
        {
            cout<<ans;
            return 0;
        }
    }
    cout<<"不存在最小生成树";	
    return 0;
}
        \end{lstlisting}
        两种算法运行结果如下:
        \begin{figure}[htbp]
            \begin{minipage}[t]{0.45\linewidth}
            \centering
            \includegraphics[height=14cm,width=8cm]{Prim.png}
            \caption{$Prim$算法}
            \end{minipage}%
            \begin{minipage}[t]{0.45\linewidth}
            \centering
            \includegraphics[height=14cm,width=8cm]{Kruskal.png}
            \caption{$Kruskal$算法}
            \end{minipage}
            \end{figure}
       % \end{figure}
       % \begin{figure}[H]
       %     \begin{tabular}
        %    \flushleft 
        %    \includegraphics[height=12cm,width=6cm]{Prim.png}
            %[height=4.5cm]表示高度
            %[width=9.5cm]表示宽度
            %{111.eps}表示eps格式的图片,名为111
            %\caption{pic1}
            %图片的名称
            %图片的标签,用于文章中的引用,注意到标签的数字与实际文章显示的数字可能不同
         %   \end{flushleft}
         %   \begin{flushleft}
         %   \flushright 
         %   \includegraphics[height=12cm,width=6cm]{Kruskal.png}
            %[height=4.5cm]表示高度
            %[width=9.5cm]表示宽度
            %{111.eps}表示eps格式的图片,名为111
            %\caption{pic1}
            %图片的名称
            %图片的标签,用于文章中的引用,注意到标签的数字与实际文章显示的数字可能不同
          %  \end{flushleft}
        
       % \end{figure}
\newpage
\section{$Prim$}
$Prim$算法中利用了线性表(链式前向星)来存储图,利用了贪心策略,每次选择找出距离集合S最近的未处理的点
进行处理并将其加入S,该边进入最小生成树,循环操作直到边数为点数$-1$生成最小生成树。

\begin{lstlisting}[language=C++,numbers=left,breaklines=true]
struct node 
{
    int to,val,next;
}ma[1000010];
void add(int a,int b,int c)//通过链式前向星加边
{
    ma[tot].to=b;
    ma[tot].val=c;
    ma[tot].next=head[a];
    head[a]=tot++;
}
int main()
{
    int n,m;
	cin>>n>>m;
	for(int i=1;i<=m;i++)
	{
		int t1,t2,t3;
		cin>>t1>>t2>>t3;
		add(t1,t2,t3);
		add(t2,t1,t3);
	}
    for(int i=2;i<=n;i++)//先将1节点到其他点距离标记为最大值
        dist[i]=2147483647;
    for(int i=head[1];i;i=ma[i].next)//标记好1点到其他点的距离
    {
        int y=ma[i].to,z=ma[i].val;
        dist[y]=min(dist[y],z);
    }
    now=1;
    vis[1]=true;//将1点标记好并将其加入到集合S中
    while(++num<=n-1)//最小生成树边数为点数-1
    {   
        int minn=2147483647;
        now=-1;
        for(int i=1;i<=n;i++)//循环遍历找出距离集合S最近的点
        {
            if(!vis[i]&&dist[i]<minn)
            {
                minn=dist[i];
                now=i;
            }
        }
        if(now==-1) //最小生成树还未生成并且找不到最近的点了说明不存在最小生成树
            break;
        vis[now]=true;//将这个点标记好并加入到集合S中
        ans+=minn;//答案更新
        for(int i=head[now];i;i=ma[i].next)//更新与now相连且未被访问的点离集合S的距离,更新距离
        {
            int y=ma[i].to,z=ma[i].val;
            if(!vis[y]&&dist[y]>z)
                dist[y]=z;
        }
    }
    if(num==n) 
    cout<<ans;
    else cout<<"不存在最小生成树";
    return 0;
}
\end{lstlisting}
\newpage
\section{分发饼干}
对于每个孩子,其能获得的饼干是那些尺寸大于等于他们胃口值的饼干,为了满足越多数量的孩子
,所以对于每个孩子应该给他们分配最小的能满足他们胃口值的饼干,使更大的饼干
能满足其他孩子的需求,这样就能使满足的孩子最多。
\begin{lstlisting}[language=C++,numbers=left,breaklines=true]
class Solution {
    public:
        int findContentChildren(vector<int>& g, vector<int>& s) {
            int n=g.size(),m=s.size();
            bool vis[30005];
            for(int i=0;i<m;i++)
            vis[i]=false;
            sort(s.begin(),s.end());//排序饼干尺寸,使每个人都尽量挑选更小的饼干
            int ans=0;
            for(int i=0;i<n;i++)
            {
                for(int j=0;j<m;j++)
                {
                    if(s[j]>=g[i]&&!vis[j])//挑选满足要求的最小的饼干
                    {
                        vis[j]=true;//标记已分配
                        ans++;
                        break;
                    }
                }
            }
            return ans;
        }
    };
    \end{lstlisting}
    \begin{figure}[H]
        \begin{flushleft}
        %\centering表示居中
        \includegraphics[height=6cm,width=16cm]{分发饼干.png}
        %[height=4.5cm]表示高度
        %[width=9.5cm]表示宽度
        %{111.eps}表示eps格式的图片,名为111
        %\caption{pic1}
        %图片的名称
        %图片的标签,用于文章中的引用,注意到标签的数字与实际文章显示的数字可能不同
        \end{flushleft}
    \end{figure}
\newpage
\section{无重叠区间}
这道题要求出需要移除区间的最小数量,使剩余区间互不重叠,可以反过来思考,
实际上就是使剩余区间最多,那么,现在就应该思考如何使选择的区间最多。可以
利用贪心想法,对于待处理区域L,为了使选择的区间最多,应该选择
最早结束且开始时间晚于上一场结束时间的那一个会议,此时需要处理
的区域为区域L减去会议右端点往左,此时剩下的区域,循环往复处理
,最终可使得选择区间最多,需要移除区间的最小。
\begin{lstlisting}[language=C++,numbers=left,breaklines=true]
class Solution {
    public:
        int eraseOverlapIntervals(vector<vector<int> >&intervals) {  
            sort(intervals.begin(),intervals.end(),[](const vector<int> &u, const vector<int> &v) {
                return u[1]<v[1];//按照右端点升序排序
            });
            int n=intervals.size(),nowright=intervals[0][1];//选择最早结束的会议
            int ans=1;
            for(int i=1;i<n;i++) 
            {
                if(intervals[i][0]>=nowright)//选择开始时间晚于上一场结束时间且是最早结束的
                {
                    ans++;
                    nowright=intervals[i][1];//更新结束时间
                }
            }
            return n-ans;
        }
};

\end{lstlisting}

\begin{figure}[H]
    \begin{flushleft}
    %\centering表示居中
    \includegraphics[height=6cm,width=16cm]{无重叠区间.png}
    %[height=4.5cm]表示高度
    %[width=9.5cm]表示宽度
    %{111.eps}表示eps格式的图片,名为111
    %\caption{pic1}
    %图片的名称
    %图片的标签,用于文章中的引用,注意到标签的数字与实际文章显示的数字可能不同
    \end{flushleft}
\end{figure}
\end{document}