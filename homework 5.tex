\documentclass{article}
\usepackage[UTF8]{ctex}
\usepackage{anyfontsize}
\usepackage{amsmath,mathtools}
\usepackage[a4paper, total={6in, 8in}]{geometry}
\usepackage{float}
\usepackage{graphicx}
\usepackage[ruled,linesnumbered]{algorithm2e}
\usepackage[usenames,dvipsnames]{xcolor}
\usepackage{listings,color}
\usepackage{tikz}
\definecolor{mypurple}{rgb}{186,85,211}
\definecolor{mygreen}{rgb}{0,0.6,0}
\definecolor{mygray}{rgb}{0.5,0.5,0.5}
\definecolor{mymauve}{rgb}{0.58,0,0.82}
\geometry{a4paper,left=2cm,right=2cm,top=3cm,bottom=3cm}
\lstset{
    language=C++,
    backgroundcolor=\color[RGB]{245,245,244},
    basicstyle          =   \sffamily,          % 基本代码风格
    keywordstyle        =   \color{ purple},         % 关键字风格
    commentstyle        =   \rmfamily\itshape,  % 注释的风格,斜体
    stringstyle         =   \ttfamily,  % 字符串风格
    flexiblecolumns,                % 别问为什么,加上这个
    numbers             =   left,   % 行号的位置在左边
    showspaces          =   false,  % 是否显示空格,显示了有点乱,所以不现实了
    numberstyle         =   \zihao{-5}\ttfamily,    % 行号的样式,小五号,tt等宽字体
    showstringspaces    =   false,
    captionpos          =   t,      % 这段代码的名字所呈现的位置,t指的是top上面
    frame               =   lrtb,   % 显示边框
}
\title{\textbf{算法设计与分析课程作业 作业5}}
\author{162140224 刘昊文}
\date{\today}

\begin{document}
\maketitle

\section{零钱兑换}
这道题可以看作一个完全背包问题,状态dp[i][j]表示前i个物品金额为j时需要的最少的硬币个数,而这个状态只能由
dp[i-1][j-coins[i]]+1或dp[i-1][j]状态转移得来
所以易得状态转移方程

\begin{equation}
    dp[i][j]=max(dp[i-1][j-k \times coins[i]]+k\times coins[i],dp[i-1][j])  \qquad k\leq \frac{j}{coins[i]}
    \nonumber
\end{equation}


这道题可以减去物品的纬度,压缩一维
但需要注意的是在枚举内层当前状态金额时,
需要从coins[i]枚举到总金额,进行正向状态转移,
使得当前状态可以利用当前这一层的状态,这代表硬币能重复拿取,符合题意

一维状态转移方程如下
\begin{equation}
    dp[j]=min(dp[j-coins[i]]+1,dp[j])
    \nonumber
\end{equation}


\begin{figure}[H]
    \begin{flushleft}
     %\centering表示居中
    \includegraphics[height=6cm,width=16cm]{零钱兑换.png}
    % [height=4.5cm]表示高度
    %[width=9.5cm]表示宽度
    %{111.eps}表示eps格式的图片,名为111
    %\caption{pic1}
    %图片的名称
    %\label{2}
    %图片的标签,用于文章中的引用,注意到标签的数字与实际文章显示的数字可能不同
    \end{flushleft}
\end{figure}


\newpage
\section{分割等和子集}
这道题需要将一个数组分为两个和相等的子数组,假设这两个子数组为num1和num2,数组总和为tot


两个子数组和需要相等,易得这两个子数组和都要为tot/2。

易得当tot为奇数时,不可能成立
当tot为偶数时,可转换为01背包问题,判断能否凑成tot/2,原问题得解


在01背包问题中,递归方程为
\begin{equation}
    dp[i][j]=max(dp[i-1][j-nums[i]]+nums[i],dp[i-1][j])
\nonumber
\end{equation}


事实上,我们可以减去物品的纬度,将其压缩至一维
递归方程如下
\begin{equation}
    dp[j]=max(dp[j],dp[j-nums[i]]+nums[i])
\nonumber
\end{equation}


需要注意的是,此时外层i循环不同物品时,内层j需要倒序循环,改变状态更新次序,
使当前状态不可以利用当前这一层的状态,防止一个数字多次被拿,与第一题正好相反
\begin{figure}[H]
    \begin{flushleft}
     %\centering表示居中
    \includegraphics[height=6cm,width=16cm]{分割等和子集.png}
    % [height=4.5cm]表示高度
    %[width=9.5cm]表示宽度
    %{111.eps}表示eps格式的图片,名为111
    %\caption{pic1}
    %图片的名称
    %\label{2}
    %图片的标签,用于文章中的引用,注意到标签的数字与实际文章显示的数字可能不同
    \end{flushleft}
\end{figure}






\newpage


\section{最长回文子序列}
这道题需要求出序列中的最长回文子序列,通过观察可得,对于一个序列,最长回文子序列等于它与它的逆序列的最长公共子序列


因此这题可以转换成一个求最长公共子序列的问题

对于求最长公共子序列,状态dp[i][j]表示当前第一个序列从1到i,第二个序列从1到j,最长回文子序列
的长度
所以容易得到状态转移方程为
\begin{equation}
dp[i][j] = \begin{cases}
dp[i-1][j-1]+1 & \text{if }  s1[i]=s2[j]\\
max(dp[i-1][j],dp[i][j-1]) & \text{if } s1[i] \neq s2[j],\\

\end{cases}
\nonumber
\end{equation}

原问题得解

\begin{figure}[H]
    \begin{flushleft}
    \includegraphics[height=6cm,width=16cm]{最长回文子序列.png}
  \end{flushleft}
\end{figure}
\newpage
\section{打家劫舍}


在这道题中,对于一个房屋,只有它旁边的房子未被闯入时
,才能闯入该房间


对于状态$dp[i][0]$ 第一维i表示前i个房间,第二维0表示第i个房间未被闯入,1表示被闯入


当第i个房间未被闯入时,第i-1个房间是否闯入都符合题意,所以选择其中较大值即 $max(dp[i-1][1],dp[i-1][0])$


当第i个房间被闯入时,第i-1个房间只能是未被闯入,所以取前i-1个房间未被闯入的值加上第i个房间的金额即$dp[i-1][0]+nums[i]$


所以易得状态转移方程
\begin{equation}
    \begin{cases}
        dp[i][1]=dp[i-1][0]+nums[i] & \text{}\\
        dp[i][0]=max(dp[i-1][1],dp[i-1][0]) & \text{} \\
    
    \end{cases}
    \nonumber
\end{equation}


\begin{figure}[H]
    \begin{flushleft}
     %\centering表示居中
    \includegraphics[height=6cm,width=16cm]{打家劫舍.png}
    % [height=4.5cm]表示高度
    %[width=9.5cm]表示宽度
    %{111.eps}表示eps格式的图片,名为111
    %\caption{pic1}
    %图片的名称
    %图片的标签,用于文章中的引用,注意到标签的数字与实际文章显示的数字可能不同
    \end{flushleft}
\end{figure}



\newpage
\section{乘积最大子数组}
这道题需要找到乘积最大的子数组,数组中即存在正数,也存在负数。所以存在乘积为负数的时候
,此时可能会负负得正得出更大值得情况
所以需要维护出子数组的最大值与最小值,设$dp[i][0]$表示第i个元素结尾的乘积最大子数组
$dp[i][1]$表示第i个元素结尾的乘积最小子数组
dp[i][1]和dp[i][0]可由dp[i-1][0]和dp[i-1][1]得到,除此之外,还可以不选择前i-1个元素,只选择
num[i],所以可得状态转移方程如下:
\begin{equation}
    \begin{cases}
        dp[i][1]=max(nums[i],max(dp[i-1][1]*nums[i],dp[i-1][0]*nums[i])) & \text{}\\
        dp[i][0]=min(nums[i],min(dp[i-1][1]*nums[i],dp[i-1][0]*nums[i])) & \text{} \\
    
    \end{cases}
    \nonumber
\end{equation}
\begin{figure}[H]
    \begin{flushleft}
     %\centering表示居中
    \includegraphics[height=6cm,width=16cm]{乘积最大子数组.png}
    % [height=4.5cm]表示高度
    %[width=9.5cm]表示宽度
    %{111.eps}表示eps格式的图片,名为111
    %\caption{pic1}
    %图片的名称
    %图片的标签,用于文章中的引用,注意到标签的数字与实际文章显示的数字可能不同
    \end{flushleft}
\end{figure}


\newpage
\section{最大子数组和}

这道题需要找出和最大的子数组,状态dp[i]表示从1到i最大的连续子数组和,易得当dp[i]大于0时
则其对后面的答案有贡献

而当dp[i]小于0时,则不计算其的负贡献,所以易得递归方程如下
\begin{equation}
    dp[i]=
    \begin{cases}
    0 & \text{if } dp[i-1] < 0,\\
    dp[i-1]+num[i] & \text{if } dp[i-1] > 0,\\
    \end{cases}
    \nonumber
\end{equation}


事实上,观察递归方程,我们可以通过一个变量tot来遍历整个数组,遍历整个数组tot=tot+num[i]
当tot<0时则将tot清零,否则继续循环
\begin{figure}[H]
    \begin{flushleft}
     %\centering表示居中
    \includegraphics[height=6cm,width=16cm]{最大子数组和.png}
    % [height=4.5cm]表示高度
    %[width=9.5cm]表示宽度
    %{111.eps}表示eps格式的图片,名为111
    %\caption{pic1}
    %图片的名称
    %图片的标签,用于文章中的引用,注意到标签的数字与实际文章显示的数字可能不同
    \end{flushleft}
\end{figure}



\newpage
\section{三角形最小路径和}
对于这道题,从上往下走,分析较为复杂,考虑从下往上走

一个点只能来自其左下与右下的点即triangle[i+1][j]与triangle[i+1][j+1]
即
\begin{center}
$triangle[1][1]=triangle[1][1]+min(triangle[2][1],triangle[2][2])$
$triangle[2][1]=triangle[2][1]+min(triangle[3][1],triangle[3][2])$
$triangle[2][2]=triangle[2][2]+min(triangle[3][2],triangle[3][3])$\\
......
\end{center}


可推出方程triangle[i][j]=triangle[i][j]+min(triangle[i+1][j],triangle[i+1][j+1])

状态转移方程如下
\begin{equation}
    triangle[i][j]=triangle[i][j]+min(triangle[i+1][j],triangle[i+1][j+1])
    \nonumber
\end{equation}


遍历方向为:从下到上从左到右
\begin{figure}[H]
    \begin{flushleft}
     %\centering表示居中
    \includegraphics[height=6cm,width=16cm]{三角形最小路径和.png}
    % [height=4.5cm]表示高度
    %[width=9.5cm]表示宽度
    %{111.eps}表示eps格式的图片,名为111
    %\caption{pic1}
    %图片的名称
    %图片的标签,用于文章中的引用,注意到标签的数字与实际文章显示的数字可能不同
    \end{flushleft}
\end{figure}

\newpage
\section{最长递增子序列}


记dp[i]表示对于前i个数,在选择第i个数的情况下,得到最长的单调不升子序列的长度,


当第i个数是子序列的第一项,dp[i]=1


第i个数不是子序列的第一项,选择的第i个数,假设其前一项为第j个数,即第i个数接在第j个数结尾的序列后面
枚举这样的j,可以得到状态转移方程:dp[i]=dp[j]+1
\begin{equation}
    dp[i]=max(dp[i],dp[j]+1)  \qquad j<i \qquad if (nums[i]>nums[j]) 
    \nonumber
\end{equation}

上述算法时间复杂度为O($n^2$)


事实上存在着O($nlog(n)$)的解法
\begin{lstlisting}[language=C++,numbers=left,breaklines=true]
class Solution {
public:
    int lengthOfLIS(vector<int>& nums) {
        int lengt=nums.size();
        int len=0,dp[2501];
        dp[0]=nums[0];
        for(int i=1;i<lengt;i++)
        {
            if(dp[len]<nums[i]) dp[++len]=nums[i];
            else dp[lower_bound(dp,dp+len,nums[i])-dp]=nums[i];
        }
        return len+1;
    }
};   
\end{lstlisting}
O($n^2$)解法耗时如下
\begin{figure}[H]
    \begin{flushleft}
     %\centering表示居中
    \includegraphics[height=6cm,width=16cm]{最长递增子序列.png}
    % [height=4.5cm]表示高度
    %[width=9.5cm]表示宽度
    %{111.eps}表示eps格式的图片,名为111
    %\caption{pic1}
    %图片的名称
    %图片的标签,用于文章中的引用,注意到标签的数字与实际文章显示的数字可能不同
    \end{flushleft}
\end{figure}
O($nlog (n)$)解法耗时如下
\begin{figure}[H]
    \begin{flushleft}
     %\centering表示居中
    \includegraphics[height=6cm,width=16cm]{最长递增子序列2.png}
    % [height=4.5cm]表示高度
    %[width=9.5cm]表示宽度
    %{111.eps}表示eps格式的图片,名为111
    %\caption{pic1}
    %图片的名称
    %图片的标签,用于文章中的引用,注意到标签的数字与实际文章显示的数字可能不同
    \end{flushleft}
\end{figure}

\end{document}